\documentclass[11pt,a4paper]{article}

\usepackage{xeCJK}
\usepackage{amsmath}
\usepackage{amssymb}
\usepackage{fontspec}
\usepackage{tikz}
\usepackage{xcolor}
\usepackage{geometry}
\usepackage{pgf,tikz}
\usepackage{clrscode}
\usepackage{enumerate}
\usetikzlibrary{shapes,arrows,automata}

\setmainfont{Times New Roman}
\setCJKmainfont{Microsoft YaHei}

\title{密码学复习题及答案 ver 1.0}
\author{Jer}

\begin{document}
\maketitle
%\tableofcontents

\section{密码学的基本安全问题是什么?公钥加密方案必须抵抗的攻击类型有哪些?}
\subsection{基本安全问题}
\subsubsection{机密问题}这指的是除了信息的授权人可以拥有信息以外,其他人都不可获得信息内容。在密码学中,主要是通过加密和解密算法来完成这项任务。
\subsubsection{数据真实完整问题}这一问题的提出是为了发现对数据的非法变更。为了做到这一点,必须提供发现非授权人对数据变动的机制。许多密码工具可以提供这一机制,如Hash函数等。
\subsubsection{认证问题}这是一个与识别相关的问题,可以应用于实体也可以应用于信息本身。两方在进行通信之前,一般需要识别对方的身份。而在信道上传输的一条信息也需要识别它是何时、何地、何内容由何人发出。因此,认证在密码学中常常被分成两类:实体认证和数据源认证。可以看出数据源认证隐含了提供数据真实性服务,这是因为数据被修改,数据源也就自然发生了变更。
\subsubsection{不可否认问题}这一问题的提出是为了阻止实体否认从前的承诺或行为。争议的发生常常是由于实体否认从前的某个行为。例如,一个实体与另一个实体签署了购买合同,但事后又否认签署过,这时常常需要一个可信第三方来解决争议。这就需要提供必要的手段来解决争议。在密码学中,解决这一问题的手段常常是数字签名。

\subsection{公钥必须抵抗的攻击类型}
\subsubsection{}(适应性)选择明文攻击
\subsubsection{}(适应性)选择密文攻击
\subsubsection{}已知明文攻击
\subsubsection{}唯密文攻击

\section{扩展Euclidean算法计算最大公约数(a, b)以及整数x和y满足(a, b)=ax+by的过程,这里a和b都是整数。如何应用费马小定理计算$2^{1000000}$模$19$的最小正整数。如何应用中国剩余定理计算同余组。群、环、域的基本概念。}

\section{几种提高DES安全强度的方法。修改发现码(MDC)的性质有哪些?}
\subsection{提高DES安全强度的方法}
\begin{enumerate}[1)]
\item 双重DES加密是使用一个密钥加密明文接着再用另一个不同的密钥加密。Merkle和Hellman使用中间人攻击表明双重DES 加密与57比特而不是112比特的安全强度相当。
\item 三重DES加密的安全强度大约可以达到112比特。至少有两个版本的三重DES加密执行,一个是:
$$ c = E_{k_1}(E_{k_2}(E_{k_3}(m))), m=D_{k_3}(D_{k_2}(D_{k_1}(c)))$$
另一个是:
$$ c = E_{k_1}(D_{k_2}(E_{k_1}(m))), m=D_{k_1}(E_{k_2}(D_{k_1}(c))) $$
这两个版本都可以抵抗中间人攻击。
\item 另一个版本的DES加密方法由Rivest提出
$$ c=K_3 \oplus E_{k_2}(K_1\oplus m), m=D_{K_2}(K_3 \oplus c) \oplus K_1$$
这一方法也叫做DESX,已经证明了其有相当的安全强度。DESX已经自1986年起被用于MailSafe电子邮件安全系统,自1987年起用于BSAFE工具包。\\
\# 这个版本的好处在于它能够很容易地在现有DES硬件上执行。
\end{enumerate}
\subsection{修改发现码(MDC)的性质}
(1) 原像不可逆:对于几乎所有的Hash输出不可能计算出其的Hash输入。也就是,在不知道输入的情况下给定任意一个输出y,找到任意一个输入x’满足h(x’)=y 是计算不可能的。
(2) 二次原像不可逆:对于任何一个给定的输入x,找到另一个输入$x’\ne x$,且满足h(x)=h(x’),在计算上不可能。
(3) 抵抗碰撞:找到两个不同的输入x和x’,满足$h(x)=h(x’)$,在计算上不可能(注意:这里两个输入可以自由选择)。

\section{AES的层有哪些? 典型的加密模式有哪些?}

\subsection{AES的层}
\begin{enumerate}
\item 字节转换
\item 移动行变换
\item 混合列变换
\item 轮密钥加密变换
\end{enumerate}
\subsection{加密模式}
\begin{enumerate}
\item 电码本模式
\item 密码分组链接模式
\item 密码反馈模式
\end{enumerate}
\section{RSA公钥加密算法及正确性证明。模4余3型素数的Rabin算法解密技术。}

\section{ElGamal加密算法及正确性证明。}

\section{RSA数字签名算法及正确性证明。}

\section{Gordon强素数生成算法及正确性证明。非邻接表(NAF)表示。}

\section{电子现金的安全要求有哪些?}
\subsection{独立}电子现金的安全不依赖任何物理地点。现金可以在计算机网络上传输。
\subsection{安全}电子现金不会被复制和重用。
\subsection{隐私(不可追踪)}用户的隐私可以得到保护,也就是,没有任何人可以追踪用户和他的交易之间的关联。
。
\subsection{离线支付}当用户为交易支付电子现金的时候,在整个交易过程中并不需要直接与银行通信。

\subsection{可转让}电子现金可以转移给其他用户。
\subsection{可分割}一个给定数量的电子现金可以被分割成更小数量的电子现金。
\section{基本的Shamir门限方案与性质。}

\subsection{Shamir的(t,n)门限方案}
\subsubsection{建立秘密} 可信方T有一个秘密$S\ge0$并希望分给n个用户。
\begin{enumerate}[1)]
\item $T$选择一个素数$p > max(S,n)$,并定义$a_0 = S$.
\item $T$选择$t-1$个随机相互独立的系数$a_1,a_2,\dots,a_{t-1}$,$0 \le a_j \le p-1$,
这样可以定义一个在$Z_p$上的随机多项式:$f(x)= \sum_{j=0}^{t-1}a_j· x^j$
\item $T$计算$S_i = f(i)(mod p) , 1 \le i \le n$(或者任意$n$个不同的点$i$,$1 \le i \le p-1$),并且安全的传递分享$S_i$以及相应的公开指标$i$给用户$P_i$。
\end{enumerate}

\subsubsection{恢复秘密} 任何$t$或更多个用户提交他们的分享。他们的分享提供了$t$个不同的点$(x,y)=(i, S_i)$, 通过$Lagrange$插值法,可以计算出所有多项式$f(x)$的系数$a_j$,$1 \le j \le t-1$,这样秘密就是$f(0)=a_0=S$。
\subsection{性质}
\subsubsection{完备} 给出任意$t-1$或更少的分享,所有共享的秘密取值$0\le S \le p-1$有相等的可能性
\subsubsection{理想}分享的数据长度与秘密长度相等。
\subsubsection{对新用户的扩展}新的分享(给新用户)可以容易的计算分配并且不影响现有的用户。
\subsubsection{多种层次控制}假如单个用户有多个密码分享,其就有相对只有单个秘密分享的用户更多的分享秘密能力,而这样的安排不会影响方案恢复秘密机制。
\subsubsection{无不能证明的假设}不同于许多密码方案,该方案的安全性不依赖于任何未经证明的假设(例如:数论困难问题)。
\section{公平电子投币协议的安全要求是什么?如何建立一个基于平方根的公平电子投币协议。}

\subsection{要求}
\begin{enumerate}[(1)]
\item Bob必须在听到Alice猜测之前就已经投币。
\item Bob不能够在听到Alice猜测之后重复投币。
\item Alice不能在其猜测之前得到投币结果。
\end{enumerate}
\subsection{基于平方根的公平电子投币协议}
\begin{enumerate}
\item Alice选择两个大的随机素数$p$和$q$,都为模4余3型。她将$p$和$q$保密,而将$n=p*q$发给Bob。
\item Bob随机选择一个整数$x$并计算$y \equiv x^2(mod n)$。他将$x$保密但发送$y$给Alice。
\item Alice使用她的$p$和$q$计算4个$y$模$n$的平方根$\pm a,\pm b$。她任意选择一个,假定为b,并发给Bob。
\item 如果$b\equiv \pm x(mod n)$,Bob告诉Alice她赢。如果$b \not\equiv\pm x(mod n)$,Bob赢。
\end{enumerate}
\section{Schnorr鉴别方案}

\subsection{系统参数选择}
\subsubsection{}选择一个适当的素数$p$满足$p-1$可以被另一个素数$q$整除。(保证模$p$的离散对数在计算上不可能。)
\subsubsection{}选择一个乘法阶为$q$的元$\beta, 1\le \beta \le p-1$。
\subsubsection{}每一方得到一份真实的系统参数($p,q,\beta$)和信任中心T的验证公开密钥的函数,它可以验证T 对消息$m$的签名$S_t(m)$。($S_t$涉及在签名之前的一个适当的公开的Hash函数和任意一个签名机制。)
\subsubsection{}选择一个安全参数$t$(例如,$t \ge40$),$2^t < q$(定义一个安全等级$2^t$)。

\subsection{每个用户参数选择}
\subsubsection{}每个实体$A$选择一个唯一身份$I_A$。
\subsubsection{}$A$选择一个秘密密钥$a$,$0 \le a \le q-1$, 并计算$v \equiv \beta^{-\alpha} (mod\ p)$。
\subsubsection{}$A$向$T$通过传统方式(例如,出示护照) 验证自己的身份,接着将真实的$v$传递给$T$,最后得到一个由$T$颁发的证书$cert_A=(I_A,v,S_T(I_A,v))$将$I_A$和$v$绑定。

\subsection{协议执行} A按如下步骤向B验证自己的身份。
\subsubsection{}A选择一个随机数$r$(承诺),$1\le r \le q-1$, 计算(证据)$x\equiv \beta^r(mod\ p)$, 并发送($cert_A, x$)给B。
\subsubsection{}B通过T在证书$cert_A$中的签名验证A的公开密钥$v$的真实性,接着发送一个(从未使用过的)随机数$e$(提问), $1 \le e \le 2^t$, 给A。
\subsubsection{}A检查$1\le e \le2^t$并发送(回答)$y\equiv a·e+r(mod\ q)$给B。
\subsubsection{}B计算$z\equiv \beta^y·v^e(mod\ p)$, 如果 $z=x$, 接受A的身份。
\section{密钥协商中的站对站(STS)协议。密钥协商协议的基本安全要求有哪些?}

\subsection{STS协议}
\subsubsection{}符号$E$表示对称加密算法。 $S_A(m)$表示Alice对消息$m$的签名。
\subsubsection{一次性的建立过程(定义和公开系统参数)}
\begin{enumerate}
\item 选择一个适当的素数$p$和一个模$p$的生成元$\alpha$并将它们公开。
\item 每个实体分别选择RSA公开和秘密签名密钥,Alice选择($e_A$,$n_A$)和$d_A$,Bob也有类似密钥对。假定每一方都可以访问对方真实的公开密钥。
\end{enumerate}
\subsubsection{协议执行} 做如下步骤以实现双方共享认证密钥(任何签名过程失败将导致协议直接以失败告终)
\begin{enumerate}
\item Alice选择一个随机秘密数字$x$,$1\le y \le p-2$,计算并发送$\alpha^x(mod p)$给Bob

\item Bob产生一个随机秘密数字 $y$, $1\le y \le p-2$,计算共享密钥$K\equiv (\alpha^x)^y(mod p)$.Bob 对两个模幂的连续签名 再用共享密钥加密, 最后发送 $\alpha^y(mod p), E_K(S_B(\alpha^y, \alpha^x))$ 给Alice。
\item Alice共享密钥$K\equiv (\alpha^y)^x(mod p)$, 解密加密数据$E_K(S_B(\alpha^y, \alpha^x))$ 并为Bob 的公开密钥验证签名。如果验证成功,Alice接受K确实为与Bob的共享密钥,最后发送 类似的$(E_K(S_A(\alpha^x, \alpha^y))$给Bob
\item Bob也类似解密收到的$E_k(S_A(\alpha^x, \alpha^y))$并验证Alice的签名。如果成功,Bob接受K 确实为与Alice的共享密钥。
\end{enumerate}
\subsection{密码协商协议的基本安全要求}
\subsubsection{已知密钥安全} 每次对密钥协商协议的运行都产生一个唯一的秘密密钥。这些密钥希望可以限制进行密码分析所能得到的数据数量,也希望能够限制密钥泄露带来的秘密数据泄露数量。协议应该达到这一安全目标即使攻击者已经掌握了一些之前的会话密钥。
\subsubsection{前项安全} 如果一个或多个实体的长期秘密密钥泄露,以前由诚实实体建立的会话密钥不受影响。有时我们区分一个实体长期秘密密钥泄露(半前项秘密)和参与双方实体长期秘密密钥泄露(全前项秘密)两种情况。
\subsubsection{密钥泄露冒充} 假定Alice的长期秘密密钥泄露。很明显,攻击者知道这个密钥可以冒充Alice,因为其确切掌握标定Alice身份的数据。但是,在某些情况下,我们希望这一泄露不能够让攻击者冒充其他实体欺骗Alice。
\subsubsection{未知共享密钥攻击} 实体Alice结束协议执行后相信她与Bob共享密钥,虽然这是实际情况,但是Bob却错误的认为他与实体 Eve$\ne$Alice共享密钥。

\end{document}
